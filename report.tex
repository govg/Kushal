% LLNCStmpl.tex
% Template file to use for LLNCS papers prepared in LaTeX
%websites for more information: http://www.springer.com

\documentclass{llncs}
%Use this line instead if you want to use running heads (i.e. headers on each page):
%\documentclass[runningheads]{llncs}


\begin{document}
\title{Popular Matchings: A survey of the state of the art}

%\subtitle{Subtitle Goes Here}

%For a single author
%\author{Author Name}

%For multiple authors:
\author{Kushal Yarlagadda \inst {1} Govind Gopakumar\inst{1}}


%If using runnningheads you can abbreviate the author name on even pages:
%\authorrunning{abbreviated author name}
%and you can change the author name in the table of contents
%\tocauthor{enhanced author name}

%For a single institute
%\institute{Institute Name \email{email address}}

% If authors are from different institutes 
\institute{Indian Institute of Technology, Kanpur \email{kushaly@iitk.ac.in}}

%to remove your email just remove '\email{email address}'
% you can also remove the thanks footnote by removing '\thanks{Thank you to...}'


\maketitle

\begin{abstract}
In this paper, we aim to provide a general outline of our study of Popular Matchings. We begin by describing the evolution of this field, from studies of matchings, to generalizations, and the study of certain aspects of Popular Matchings. We then outline a certain kind of matchings, and the actions possible by agents involved in the matchings \ldots\ldots
\end{abstract}

\section{Introduction}
\subsection{Matchings to Popular Matchings}

In this paper, we discuss  matchings in a bipartite graph G(V,E), with V $=$ A\cupP, and E $=$ (a,b):\lbrace a $\in$ A, b $\in$ P\rbrace. 
For every one of the vertices in A, corresponding to an Applicant, we have a preference list $P_a$,
which denote the posts to which he/she would like to be matched, and ordered according to their preference betweent the various posts. In the graph G, this corresponds to edges between vertices in A and those in P. 
Consider the preference lists across all applicants in A. 
We construct a subset of edges, namely, $E_1$ corresponding to all the edges which have been assigned preference/rank 1 by any of the applicants. 
This construction can be extended to all the edges in the graph, thus, we may have subsets denoted by $E_2, E_3, \cdots$ 
Between the posts preferred by a particular applicant, the subscript of the edge denotes the rank/preference of the post. So, $e_1$ would be a edge/post preferred more than $e_2$.
We call a preference list strict, when for each preference, only one post is listed. Correspondingly, for each vertex in A, there is only on edge which has the first preferece, only one which has the second, and so on. The compliment to this is the case of ties. In the case of preferences with ties, more than one post may be listed as being the $n^th$ preference of any of the applicants.


An applicant, therefore, can be said to prefer a matching $M_1$ to a matching $M_2$ if he prefers the post assigned in $M_1$ to his post in  $M_2$.  

A matching $M_1$ is said to \emph{more popular} than matching $M_2$ is the number of applicants who prefer $M_1$ to $M_2$ are strictly more than those who prefer $M_2$ to $M_1$ .  
Note that some of the applicants may be indifferent between the two matchings 

























\bibliographystyle{splncs}

%the following does the same as above except with alphabetic sorting
%\bibliographystyle{splncs_srt}
%the following is the current LNCS BibTex with alphabetic sorting
%\bibliographystyle{splncs03}
%If you want to use a different BibTex style include [oribibl] in the document class line

\begin{thebibliography}{1}
%add each reference in here like this:
\bibitem[RE1]{reference1}
Author:
Article/Book:
Other info: (date) page numbers.
\end{thebibliography}

\end{document}
