% LLNCStmpl.tex
% Template file to use for LLNCS papers prepared in LaTeX
%websites for more information: http://www.springer.com
%http://www.springer.com/lncs

\documentclass{llncs}
%Use this line instead if you want to use running heads (i.e. headers on each page):
%\documentclass[runningheads]{llncs}


\begin{document}
\title{Popular Matchings: A survey of the state of the art}

%\subtitle{Subtitle Goes Here}

%For a single author
%\author{Author Name}

%For multiple authors:
\author{Kushal Yarlagadda \inst {1} Govind Gopakumar\inst{1}}


%If using runnningheads you can abbreviate the author name on even pages:
%\authorrunning{abbreviated author name}
%and you can change the author name in the table of contents
%\tocauthor{enhanced author name}

%For a single institute
%\institute{Institute Name \email{email address}}

% If authors are from different institutes 
\institute{Indian Institute of Technology, Kanpur \email{kushaly@iitk.ac.in}}

%to remove your email just remove '\email{email address}'
% you can also remove the thanks footnote by removing '\thanks{Thank you to...}'


\maketitle

\begin{abstract}
In this paper, we aim to provide a general outline of our study of Popular Matchings. We begin by describing the evolution of this field, from studies of matchings, to generalizations, and the study of certain aspects of Popular Matchings. We then outline a certain kind of matchings, and the actions possible by agents involved in the matchings \ldots\ldots
\end{abstract}

\section{Introduction}
\subsection{Matchings to Popular Matchings}

In this paper, we discuss  matchings in a bipartite graph G, with vertices $\in$ A\cupP, and edges of the form (a,b):\lbrace a $\in$ A, b $\in$ P\rbrace. For every one of the vertices in A, corresponding to an Applicant, we have a preference list $P_a$, which contains a rank assignment to every one of the edges incident on it. These lists make divide the edges of the graph into sets according to their preferences, $E_1$ denoting the set of all edges which have been given rank 1 by any of the vertices neighbouing it. This holds for the corresponding edge sets $E_2, E_3 \cdots E_n$. These preference lists indicate the choice of each of the Applicants, in the set A. When we say that an applicant A prefers edge $e_a$ to edge $e_b$, we mean that $e_a$ appears higher than $e_b$  in the preference list of A, or that $b \> a$.  



























\bibliographystyle{splncs}

%the following does the same as above except with alphabetic sorting
%\bibliographystyle{splncs_srt}
%the following is the current LNCS BibTex with alphabetic sorting
%\bibliographystyle{splncs03}
%If you want to use a different BibTex style include [oribibl] in the document class line

\begin{thebibliography}{1}
%add each reference in here like this:
\bibitem[RE1]{reference1}
Author:
Article/Book:
Other info: (date) page numbers.
\end{thebibliography}

\end{document}
